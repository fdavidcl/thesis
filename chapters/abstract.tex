% \pagelayout{margin}%
\chapter*{Abstract}

\begin{center}
\textbf{Improvements in treatment of classification problems with autoencoder-based models}
\end{center}

The present doctoral thesis tackles the study and application of a specific tool from the data science field, autoencoders, which are artificial neural networks able to transform the variable space of a dataset according to a certain selected criterion. Manipulating and transforming variables is a crucial task in data mining, since it can largely determine the complexity of a data analysis problem and, as a result, affect the behavior of learning methods which are used to extract useful knowledge. Moreover, the recent surge in data collection and processing for all kinds of purposes causes these tasks to be less and less feasible to be performed by hand, so there is a need of automatic methods to solve them.

Autoencoders are models that belong to the field of representation learning, and can be much more flexible and adaptable than other, more classical methods such as principal component analysis. This versatility has been studied via a thorough analysis of their inner workings and the different varieties of models than can be created based on their essential components. As a complement, a new software tool has been developed to provide easy access to these models and eliminate an important existing knowledge barrier which could prevent their use.

An extensive search has been conducted in the literature for problem typologies whose difficulty is related to the data representation, so as to open the possibility for autoencoder-based solutions. Datasets can present several issues: those linked to the very structure of the data points, like the use of several objects to describe a sole instance; those relative to the complexity of categorized data, or tasks that do not provide additional information and must be solved by means of feature analysis.

With the aim of creating a novel contribution in the field of autoencoders, three new models have been developed to tackle the problem of complexity in categorized data. They are able to simplify the borders between categories in order for a classification method to improve its performance.

In summary, the main contributions of this thesis are as follows:

\begin{itemize}
    \item A \textbf{theoretical analysis and taxonomy} of the main autoencoder variants present in the literature, composing a guide to ease their selection and use.
    \item A complete \textbf{software package} which automatizes a great part of the implementation work for autoencoders and simplifies its use to a level similar to other feature extraction methods.
    \item A \textbf{synthesis and organization} work of the peculiarities that supervised learning problems can present when data points are represented in a nonstandard fashion.
    \item A \textbf{demonstration of the diverse applications} of autoencoder-based models, identifying and exposing several strategies to solve unsupervised problems by means of variable transformations.
    \item Three new models, \textbf{Scorer, Skaler and Slicer}, focused on data complexity reduction in classification problems.
\end{itemize}

This document introduces all global concepts needed to understand the published articles and provides a theoretical vision of the representation learning problem and of the deep learning tool set, which includes the main object of study. In addition, it explains the techniques that help put into practice these models and how they execute on computation infrastructures. Next, the material published throughout the doctoral period is introduced and five articles published in renowned journals are reproduced. Finally, these and other activities carried out are summarized and the lines of work that would continue the achieved advancements are presented.

\pagelayout{wide}%