\setchapterpreamble[u]{\margintoc}
\chapter{Introduction}
\labch{intro}


The current trends in collection of data are increasing, with more and more human activities producing machine-readable information, such as product reviews, posts on social media, medical images, industrial machinery sensor data, \xr{más ejemplos y citas} and more. Automatic processing of data makes it easier to obtain  results fast and saves hours of human labor which can be freed for other purposes or dedicated to tasks which cannot be automated. The speed provided by current computation resources also opens new possibilities for leveraging the available data, achieving extraction and manipulation of information at levels infeasible by human hand.

The study of problems, tools and solutions related to data integration, processing and analysis has been recently known as data science \xr{cita}, a field which overlaps branches of mathematics, statistics and computer science, among other disciplines. Current data science applications are present everywhere, from the most industrial settings to direct final user access, and range from machinery fault detection, to medical diagnosis assistance, customer loyalty in retail and photograph enhancing \xr{citas}.

The general objective in a data science problem is to model a real world scenario based on the collected data and use the resulting model to provide some information to the end user, for example, a category or label, a ranking, an association or a transformed version of the original data. This is a process that encompasses all steps from data acquisition, to its preparation, processing and analysis.

\section{Problem setting}

A great part of the time spent in a data science problem, usually the longest, consists in preparing and preprocessing the available data in order for the posterior learning techniques to extract the maximum possible amount of information.

One aspect of data to which machine learning \xr{definition?} models may be specially sensitive is the set of features, the specific representation of each instance.

\section{Motivation}

The questions that we are trying to tackle throughout this thesis can be summarized as follows:

\begin{itemize}
    \item How does one approach representation learning with deep neural models?
    \item What benefits can one obtain by transforming data into an appropriate representation?
    \item Can one induce specific behavior within the transformations, such as separating different classes?
\end{itemize}

- Problemas a la hora de generar modelos para los datos: problemáticas no estándares, dimensionalidad y complejidad de clases

\section{Objective}

\section{Tools}

- Qué es el aprendizaje profundo y cómo surge

- Por qué aplicar aprendizaje profundo

\section{Thesis structure}

The rest of this document contains all the needed information to develop the posed questions and provide enough context in order to understand the work that has been realized throuhgout the thesis.

\begin{enumerate}
    \item Chapter~\ref{ch:theory} describes the basic theory that lies under the later work and the specific models that are developed and used.
    \item \xr{Chapter...}
\end{enumerate}