% Load the kaobook class
\documentclass[
	fontsize=11pt, % Base font size
	twoside=true, % Use different layouts for even and odd pages (in particular, if twoside=true, the margin column will be always on the outside)
	open=any, % If twoside=true, uncomment this to force new chapters to start on any page, not only on right (odd) pages
	secnumdepth=1, % How deep to number headings. Defaults to 1 (sections)
]{kaobook}

\usepackage{latt}
% Choose the language
\usepackage[english]{babel} % Load characters and hyphenation
\usepackage[english=american]{csquotes}	% English quotes
\usepackage[utf8]{inputenc}
\usepackage[T1]{fontenc}

% Load packages for testing
\usepackage{blindtext}
%\usepackage{showframe} % Uncomment to show boxes around the text area, margin, header and footer
%\usepackage{showlabels} % Uncomment to output the content of \label commands to the document where they are used

% Load the bibliography package
\usepackage{kaobiblio}
\addbibresource{references.bib} % Bibliography file

% Load mathematical packages for theorems and related environments
\usepackage{kaotheorems}

% Load the package for hyperreferences
\usepackage{kaorefs}

\graphicspath{{images/}{./}{assets/}} % Paths where images are looked for


% \usepackage[fencedCode,inlineFootnotes,citations,definitionLists,hashEnumerators,smartEllipses,pipeTables,tableCaptions,hybrid]{markdown}


% \makeindex[columns=3, title=Alphabetical Index, intoc] % Make LaTeX produce the files required to compile the index
% \usepackage[refpage]{nomencl}
\renewcommand{\nomname}{List of Abbreviations}
% \nomrefpage
\makenomenclature


\begin{document}

%----------------------------------------------------------------------------------------
%	BOOK INFORMATION
%----------------------------------------------------------------------------------------

% \titlehead{}
\title[Autoencoders from scratch]{Autoencoders from scratch}
\subtitle{Notes and musings on autoencoders from a PhD student}
\author[Francisco David Charte Luque]{Francisco David Charte Luque\\\texttt{<fdavidcl@ugr.es>}}
% \department{Departamento de Ciencias de la Computación e Inteligencia Artificial}
% \supervisor{Francisco Herrera Triguero\\Francisco Charte Ojeda}
\publicationplace{Granada}
\publicationdate{15 de julio de 2022}
% \date{\underline{Directores de tesis}\\\textbf{Dr. D. Francisco Herrera}\\Soft Computing y Sistemas de Información Inteligentes (Universidad de Granada)\\\textbf{Dr. D. Francisco Charte}\\Sistemas Inteligentes y Minería de Datos (Universidad de Jaén)}
% \publishers{An Awesome Publisher}

%----------------------------------------------------------------------------------------

\frontmatter % Denotes the start of the pre-document content, uses roman numerals

%----------------------------------------------------------------------------------------
%	COPYRIGHT PAGE
%----------------------------------------------------------------------------------------

\makeatletter
\uppertitleback{\@title} % Header

\lowertitleback{
	%\textbf{Disclaimer} \\
	%You can edit this page to suit your needs. For instance, here we have a no copyright statement, a colophon and some other information. This page is based on the corresponding page of Ken Arroyo Ohori's thesis, with minimal changes.
	
	%\medskip
	
	\textbf{Funding} \\
	This doctoral thesis was funded by the predoctoral program Formación del Profesorado Universitario (ref. FPU17/04069) from the Spanish Ministry of Universities.

	\medskip

	\textbf{Copyright} \\
	This book is released under a CC-BY-SA copyright license. This allows you to share and modify it as long as the modifications are released under a similar license.

	To view a copy of the CC-BY-SA code, visit: \\\url{https://creativecommons.org/licenses/by-sa/4.0/}
	
	\medskip
	
	\textbf{Colophon} \\
	This document was typeset with the help of \href{https://sourceforge.net/projects/koma-script/}{\KOMAScript} and \href{https://www.latex-project.org/}{\LaTeX} using the \href{https://github.com/fmarotta/kaobook/}{kaobook} class.
	
	\medskip
	
	% \textbf{Publisher} \\
	% First printed in May 2019 by \@publishers
}
\makeatother

%----------------------------------------------------------------------------------------
%	DEDICATION
%----------------------------------------------------------------------------------------

\dedication{\itshape
% This is only a foretaste of what is to come, and only the shadow of what is going to be.
	Machines take me by surprise with great frequency. \\
	\flushright -- Alan Turing
}

%----------------------------------------------------------------------------------------
%	OUTPUT TITLE PAGE AND PREVIOUS
%----------------------------------------------------------------------------------------

% Note that \maketitle outputs the pages before here
\makeouterpage
\maketitle

\pagestyle{pagenum.scrheadings}

%----------------------------------------------------------------------------------------
%	PREFACE
%----------------------------------------------------------------------------------------

\chapter*{Abstract}

\blindtext

\chapter*{Resumen}

\blindtext

%----------------------------------------------------------------------------------------
%	TABLE OF CONTENTS & LIST OF FIGURES/TABLES
%----------------------------------------------------------------------------------------

\begingroup % Local scope for the following commands

% Define the style for the TOC, LOF, and LOT
%\setstretch{1} % Uncomment to modify line spacing in the ToC
%\hypersetup{linkcolor=blue} % Uncomment to set the colour of links in the ToC
\setlength{\textheight}{230\vscale} % Manually adjust the height of the ToC pages

% Turn on compatibility mode for the etoc package
\etocstandarddisplaystyle % "toc display" as if etoc was not loaded
\etocstandardlines % "toc lines as if etoc was not loaded

\tableofcontents % Output the table of contents

\listoffigures % Output the list of figures

% Comment both of the following lines to have the LOF and the LOT on different pages
\let\cleardoublepage\bigskip
\let\clearpage\bigskip

\listoftables % Output the list of tables

\endgroup

%----------------------------------------------------------------------------------------
%	MAIN BODY
%----------------------------------------------------------------------------------------

\mainmatter % Denotes the start of the main document content, resets page numbering and uses arabic numbers
\setchapterstyle{kao} % Choose the default chapter heading style

\setchapterpreamble[u]{\margintoc}
\chapter{Introduction}
\labch{intro}


The current trends in collection of data are increasing, with more and more human activities producing machine-readable information, such as product reviews, posts on social media, medical images, industrial machinery sensor data, \xr{más ejemplos y citas} and more. Automatic processing of data makes it easier to obtain  results fast and saves hours of human labor which can be freed for other purposes or dedicated to tasks which cannot be automated. The speed provided by current computation resources also opens new possibilities for leveraging the available data, achieving extraction and manipulation of information at levels infeasible by human hand.

The study of problems, tools and solutions related to data integration, processing and analysis has been recently known as data science \xr{cita}, a field which overlaps branches of mathematics, statistics and computer science, among other disciplines. Current data science applications are present everywhere, from the most industrial settings to direct final user access, and range from machinery fault detection, to medical diagnosis assistance, customer loyalty in retail and photograph enhancing \xr{citas}.

The general objective in a data science problem is to model a real world scenario based on the collected data and use the resulting model to provide some information to the end user, for example, a category or label, a ranking, an association or a transformed version of the original data. This is a process that encompasses all steps from data acquisition, to its preparation, processing and analysis.

\section{Problem setting}

A great part of the time spent in a data science problem, usually the longest, consists in preparing and preprocessing the available data in order for the posterior learning techniques to extract the maximum possible amount of information \xr{cita preprocessing}. There exist several traits of the data that can be manipulated to facilitate the work of learning algorithms: missing values, noisy instances, class imbalance, among others.

One aspect of data to which machine learning \xr{definition?} models may be specially sensitive is the set of features, the specific representation of each instance.

- what difficulties can the feature set present

Features that may characterize an event or object adequately for humans may not be ideal for machines to process. For example, a string of text may have meaning for a reader but for the machine it is just a sequence of characters. Furthermore, the techniques used for collecting data can only produce the ``observable'' variables in a dataset, but there may be interesting, ``hidden'' variables which influence the data in a clearer way.

- what problems cause the feature set on classifiers

Providing a learning algorithm with unprocessed features usually leads to sub-par performance due to 

- what is the objective, to get a better feature set

All this leads to a new task that can be performed before the actual knowledge extraction, \textit{feature learning}. The objective when learning features is 

- what are the difficulties of obtaining a new feature set

- what are other possible difficulties: difficult classes and nonstandard problems

Other possible obstacles that one may come across when inspecting the features of a dataset are difficult classes, also known as class complexity. This scenario arises when the variables do not provide sufficient information to allow class separation, the relations between variables and the target are highly nonlinear, or other circumstances prevent a learning method from inferring an adequate mapping from the input features to the target variable.

\section{Tools}

- Qué es el aprendizaje profundo y cómo surge

- Por qué aplicar aprendizaje profundo



\section{Motivation}

The questions that we are trying to tackle throughout this thesis can be summarized as follows:

\begin{itemize}
    \item How does one approach representation learning with deep neural models?
    \item What benefits can one obtain by transforming data into an appropriate representation?
    \item Can one induce specific behavior within the transformations, such as separating different classes?
\end{itemize}


% - the potential of deep neural models over other methods for feature learning

As the trends in usage of deep learning models to solve machine learning problems continue to increase, we focus our interest in their potential to not only tackle supervised problems such as classification, regression or detection, but also wider problems where solutions are not so easily validated, such as feature learning. Since deep learning allows the integration of the feature extraction stage directly within the predictor itself, these types of models should be valuable feature learners for other tasks as well.

% - interest in how to adapt feature generation for different purposes, e.g. multiple outputs

Deep neural models dedicated to generating new feature sets could be adapted, as a result, to different purposes. For instance, one could search for feature spaces where ``ordinary'' data points are very cohesive, and thus anomalous inputs would be easy to identify. Similarly, a transformation of features could allow for better separation of different classes, better distinction between noise and signal, or more meaningful traits that relate to all the original variables.

% - interest in multi-view problems

Another potential use of feature learning models that catches our interest is the possibility of capturing more than one aspect of each problem instance, which translates to different \textit{views} of the problem, for example, image and text. These views could be processed by different learners or special algorithms, but one could build feature learners that combine the available information into a more machine-ready feature vector.

% - interest in interpretable models

A current conflict of deep learning models with several areas of interest in machine learning, such as medical applications, is the fact that most are essentially black boxes. This means that the behavior of a trained model is obscured by the intrinsic structure and is thus unintuitive for humans to comprehend. As a result, there is much interest in explaining and justifying the behavior of these models, 

\section{Objective}

- explore ways of learning feature sets by means of deep learning models

- develop new objectives so that features are able to separate different classes better

\section{Thesis structure}

The rest of this document contains all the needed information to develop the posed questions and provide enough context in order to understand the work that has been realized throuhgout the thesis.

\begin{enumerate}
    \item Chapter~\ref{ch:theory} describes the basic theory that lies under the later work and the specific models that are developed and used.
    \item \xr{Chapter...}
\end{enumerate}
\setchapterpreamble[u]{\margintoc}
\chapter{Theoretical foundation}
\labch{theory}


% The main purpose of this chapter is to make it obvious for
% the reader that the report authors have made an effort to read
% up on related research and other information of relevance for
% the research questions. It is a question of trust. Can I as a
% reader rely on what the authors are saying? If it is obvious
% that the authors know the topic area well and clearly present
% their lessons learned, it raises the perceived quality of the
% entire report.

% \begin{kaobox}[frametitle=Remark]
% After having read the theory chapter it shall be obvious for
% the reader that the research questions are both well
% formulated and relevant.
% \end{kaobox}

% The chapter must contain theory of use for the intended
% study, both in terms of technique and method. If a final thesis
% project is about the development of a new search engine for
% a certain application domain, the theory must bring up related
% work on search algorithms and related techniques, but also
% methods for evaluating search engines, including
% performance measures such as precision, accuracy and
% recall.

% The chapter shall be structured thematically, not per author.
% A good approach to making a review of scientific literature
% is to use \emph{Google Scholar} (which also has the useful function
% \emph{Cite}). By iterating between searching for articles and reading
% abstracts to find new terms to guide further searches, it is
% fairly straight forward to locate good and relevant
% information, such as \cite{test}.

% Having found a relevant article one can use the function for
% viewing other articles that have cited this particular article,
% and also go through the article’s own reference list. Among
% these articles on can often find other interesting articles and
% thus proceed further.

% It can also be a good idea to consider which sources seem
% most relevant for the problem area at hand. Are there any
% special conference or journal that often occurs one can search
% in more detail in lists of published articles from these venues
% in particular. One can also search for the web sites of
% important authors and investigate what they have published
% in general.

% This chapter is called either \emph{Theory, Related Work}, or
% \emph{Related Research}. Check with your supervisor.

\section{Machine learning fundamentals}

Machine learning differs from other kinds of computer science disciplines in that its objective is not to give precise instructions for the machine to follow, but instead to provide some form of experience that the machine must learn from in order to extract some information or display some behavior \cite{deisenroth2020mathematics}. The algorithms developed for machine learning are essentially mechanisms that take in a certain amount of data, process it and compute the necessary steps to fulfill a specific objetive related to the data. Their output is usually a \textit{model}, that is, a representation of an approximate solution to the problem. 
 
\subsection{Data and models}

\begin{margintable}
\caption{\label{tbl:dataset}An example dataset describing features of different kinds of animals. Each feature can be numerical (length, legs) or categorical (wings, species).}\footnotesize
\begin{tabular}{rrrl}\toprule
Length & Legs & Wings & Species\\\midrule
40 & 4 & No & Dog\\
1 & 6 & Yes & Fly\\
145 & 0 & No & Dolphin\\\bottomrule
\end{tabular}
\end{margintable}

Datum (plural \textit{data}) usually refers to the minimal unit of machine-readable information, for example, the height of a person (numerical value), whether they are an adult or not (binary categorical value), their country of origin (categorical value) or their given name (character string).

A \textit{dataset} is a collection of data, usually organized into a table (an example is shown in \autoref{tbl:dataset}). It contains several \textit{samples}, which correspond to each one of the cases of the problem from which the machine will be able to learn before being presented with new cases. \textit{Variables} are each one of the aspects that have been measured or that characterize each sample. Samples are typically distributed in rows and variables in columns.

% \subsection{Models}

A \textit{model} is an abstraction of a dataset that enables the machine to perform the desired operations, for example, generating new data similar to the available, or assigning a category to new data points. A good model should be faithful to the available data, incorporating enough information to describe its behavior and potential relations between variables, so that it can be used as a description of the data and as a tool for solving tasks related to it. Models typically follow some template which includes a range of parameters that can be adjusted in order for the resulting model to represent the data. We will call these templates \textit{untrained models}, whereas the final results will be \textit{trained models}.

\subsection{Learning and types of learning}

In the context of machines learning from data, several types of learning are usually distinguished, according to the feedback that the machine receives while processing data. This concept is known as \textit{supervision}, and usually relates to whether there are available solved cases for the specific problem at hand. A solved case is composed of an input instance and an associated solution or \textit{label}, which may be a numerical value, a categorical value or a more complex structure.
 
\begin{itemize}
    \item Supervised learning (SL\nomenclature{SL}{Supervised learning})
    \item Unsupervised learning
    \item Semi-supervised learning
    \item Reinforcement learning
\end{itemize}

\subsubsection{Supervised learning}

In a supervised learning setting, every observed case of the problem in the dataset is coupled with its solution, so that the machine can learn a mapping out of those associations, from the space of the instances (input space) to the one of the labels (output space). In \autoref{tbl:dataset}, if the objective task is to predict the species of an animal, knowing the rest of its characteristics, this would be appropriate for a supervised learning algorithm. 

Common supervised learning problems are \textit{classification} and \textit{regression}.

\subsubsection{Unsupervised learning}

This scenario consists in problems where the solution is not known for the data that is available and, as a result, the model cannot be provided with supervision. 

\subsubsection{Semi-supervised learning}

\subsubsection{Reinforcement learning}


\section{Deep learning}

Traditional algorithms for adjusting models tend to process data "as is", which means that they perform few transformations (or none) to each vector before using them directly to fit model parameters. This causes them to underperform when the representation of the vectors (i.e. the set of features) is not ideal. As a consequence, it is usually convenient to preprocess data beforehand, using one or several tools that will manipulate the features looking to improve the performance of the learning algorithm. This is known as \textit{feature extraction}, feature learning or representation learning. 

An alternative approach is to embed the feature extraction stage within the untrained model itself, and learn the best features at the same time that the final model (a classifier, regressor, segmentor...) is trained. When this process is organized layer-wise, the overall model is called a \textit{deep learning} model.

\section{Encoder-decoder architectures}

There exists a category of deep architectures composed of two components, an \textit{encoder} and a \textit{decoder}, where there is an interest in the model operating first with the features in order to obtain higher-level features (encoding) and then developing these features back onto more detailed and specific versions.

For example, when the objective task is to segment the pixels in an image, that is, label each pixel with one of several classes, a possible solution is to compute abstract, high-level features for the image, and use those to classify each pixel next. This allows to analyze the neighborhoods of each pixel before assigning it to a class, which will probably lead to more cohesive segmentations. Using an encoder-decoder structure, the encoder would compute these low-resolution but high-level features, and the decoder would perform the detailed labeling task out of the extracted information.

A special subset of encoder-decoder architectures are autoencoders, which are further described next.

\subsection{Autoencoders}

Essentially, an \textit{autoencoder} is an encoder-decoder architecture which is trained to map its inputs onto its outputs.

% \section{}

% \refstepcounter{article}%
% \let\oldthechapter\thechapter
% \renewcommand*{\thechapter}{\thearticle}%
% \chapter{A tutorial on autoencoders}%
% \begin{widepar}
% \begin{kaobox}[frametitle=Source]
%   \fullcite{INFFUS18-AutoencoderTutorial}\\[2\baselineskip]
% \end{kaobox}
% \end{widepar}
% \includepdf[pages=-]{INFFUS18-AutoencoderTutorial}

\setchapterpreamble[u]{\margintoc}
\chapter{Practical details}
\labch{practical}

This chapter is dedicated to illustrating the reader on the different possibilities that exist for designing and implementing autoencoder networks.

\section{Design}

Designing an autoencoder for a certain task can be challenging, since the objective is to find a more useful representation of the data but we cannot know the size of the optimal representation beforehand, thus difficulting decisions about the number of layers and the size of each one.

\subsection{Type of layers}

\subsection{Model depth}

\subsection{Encoding layer}

\subsection{Output layer}


\section{Implementation}

During the latest years, there has been a notable evolution in the scene of software libraries for deep learning. From the existence of a wide variety of them with differing functionalities, ease of use and optimizations, there has been a tendency to condensate popularity in just two of them, which currently offer very similar functionalities and interfaces: Tensorflow and Pytorch.

\begin{figure*}[htbp]
    \centering  
    \includegraphics[width=\linewidth]{library_trends.png}
    \caption{Trends for web searches for five of the most popular deep learning frameworks, over the last 5 years.}
    \label{fig:trends}
\end{figure*}

\subsection{About Tensorflow}

- whodunnit
- main features
- small AE example

\subsection{About Pytorch}

- same

% \makearticle{A tutorial on autoencoders}{INFFUS18-AutoencoderTutorial}

\providecommand{\addarticle}[2]{\addcontentsline{toc}{chapter}{#1}%
\markboth{}{#1}%
\includepdf[pages=1,addtotoc={1,chapter,2,#1,p}]{#2}%
\includepdf[pages=2-,pagecommand={}]{#2}}

% \addarticle{I~~~A practical tutorial on autoencoders for nonlinear feature fusion}{papers/01-autoencoder-review/paper.pdf}

% \addarticle{II~~~Ruta: implementations of neural autoencoders in R}{papers/02-ruta/paper.pdf}

% \addarticle{III~~~A snapshot on nonstandard supervised learning problems}{papers/03-nonstandard/paper.pdf}

% \includearticle{papers/01-autoencoder-review/chapter.tex}
% \includearticle{papers/02-ruta/chapter.tex}
% \includearticle{papers/03-nonstandard/chapter.tex}
% \includearticle{papers/04-showcase/chapter.tex}
% \includearticle{papers/05-case-studies/chapter.tex}
% \includearticle{papers/06-reducing/chapter.tex}
% \includearticle{papers/07-slicer/chapter.tex}
 
% \addarticle{IV~~~A showcase of the use of autoencoders in feature learning applications}{papers/04-showcase/paper.pdf}
% \addarticle{V~~~An analysis on the use of autoencoders for representation learning}{papers/05-case-studies/paper.pdf}
% \addarticle{VI~~~Reducing data complexity using autoencoders with class-informed loss functions}{papers/06-reducing/paper.pdf}
% \addarticle{VII~~~Feature learning for class separability applied to COVID-19 chest X-ray case study}{papers/07-slicer/paper.pdf}

% \makearticle{Software for unsupervised deep learning architectures in R}{knosys-RUTA-FINAL}

% \chapter{Software for unsupervised deep learning architectures in R}%
% \begin{widepar}
% \begin{kaobox}[frametitle=Source]
%   \fullcite{knosys-RUTA-FINAL}\\[2\baselineskip]
% \end{kaobox}
% \end{widepar}
% % \includepdf[pages=-]{knosys-RUTA-FINAL}
% \renewcommand*{\thechapter}{\oldthechapter}%
% \pagelayout{margin}
 

% % \makepart{Published articles}

% % \makearticle{A tutorial on autoencoders}{INFFUS18-AutoencoderTutorial}

% % This paper tackles the variety...

% % \includepdf{INFFUS18-AutoencoderTutorial}

\setchapterpreamble[u]{\margintoc}
\chapter{Conclusions}
\label{ch:conclusions}

This chapter aims to summarize the outcomes of this thesis, highlight the most relevant achievements, list the related published material and outline some lines of work that will be pursued next.

\section{Achieved objectives and results}

\subsection{Didactic resources for learning about autoencoders}

Autoencoders are conceptually very different from traditional feature extractors. Unlike these, autoencoders are based on a neural network framework and this allows for a high level of customization and adjustments for each task. However, this availability of diverse options when building an autoencoder makes it less accessible to inexperienced practitioners. This is a barrier that was identified at the start of our research work and, as a result, become 

Our first goal, taking advantage of the usual literature review, was to produce a guide on autoencoder for machine learning users assuming no prior knowledge about neural networks or these models in particular.

\subsection{Software tool for easy access to autoencoders}

One of the first obstacles that programmers may come across when working with feature learning tools is that autoencoders are much harder to set up and train than other alternatives like PCA or even complex manifold learning algorithms such as LLE or Isomap, which come already implemented in libraries and can be applied with a simple function call.


\subsection{Development of new autoencoder losses for class separability}

\section{Summary of publications}

This section holds a relation of all public results of the thesis, including the publications that have been reproduced from \autoref{ch:paper1} to \autoref{ch:paper7}, the related software packages and repositories that allow to replicate experimental results, as well as publications arising from collaborations with colleagues and other projects.

\subsection{Publications associated to the thesis}

Following are the publications in JCR journals and international conferences associated to the present thesis.

\subsubsection{In JCR journals}

\begin{itemize}
    \item Charte, D., Charte, F., García, S., del Jesus, M. J., \& Herrera, F. (2018). A practical tutorial on autoencoders for nonlinear feature fusion: Taxonomy, models, software and guidelines. Information Fusion, 44, 78-96.
    \item Charte, D., Charte, F., García, S., \& Herrera, F. (2019). A snapshot on nonstandard supervised learning problems: taxonomy, relationships, problem transformations and algorithm adaptations. Progress in Artificial Intelligence, 8(1), 1-14.
    \item Charte, D., Herrera, F., \& Charte, F. (2019). Ruta: Implementations of neural autoencoders in R. Knowledge-Based Systems, 174, 4-8.
    \item Charte, D., Charte, F., del Jesus, M. J., \& Herrera, F. (2020). An analysis on the use of autoencoders for representation learning: Fundamentals, learning task case studies, explainability and challenges. Neurocomputing, 404, 93-107.
    \item Charte, D., Charte, F., \& Herrera, F. (2021). Reducing Data Complexity using Autoencoders with Class-informed Loss Functions. IEEE Transactions on Pattern Analysis and Machine Intelligence.
\end{itemize}

\subsubsection{In international conferences}

\begin{itemize}
    \item Charte, D., Charte, F., del Jesus, M. J., \& Herrera, F. (2019, June). A Showcase of the Use of Autoencoders in Feature Learning Applications. In International Work-Conference on the Interplay Between Natural and Artificial Computation (pp. 412-421). Springer, Cham.
    \item Charte, D., Sevillano-García, I., Lucena-González, M. J., Martín-Rodríguez, J. L., Charte, F., \& Herrera, F. (2021, September). Slicer: Feature Learning for Class Separability with Least-Squares Support Vector Machine Loss and COVID-19 Chest X-Ray Case Study. In International Conference on Hybrid Artificial Intelligence Systems (pp. 305-315). Springer, Cham.
\end{itemize}

\subsection{Published software}

\begin{itemize}
    \item Ruta, software for unsupervised deep architectures (associated to \autoref{ch:paper2}). Homepage: \href{https://ruta.software/}{ruta.software}. Source code: \href{https://github.com/fdavidcl/ruta}{github.com/fdavidcl/ruta}.
    \item autoencoder-showcase (associated to \autoref{ch:paper4}). Homepage/source code: \href{https://github.com/ari-dasci/S-autoencoder-showcase}{github.com/ari-dasci/S-autoencoder-showcase}.
    \item ae-case-studies (associated to \autoref{ch:paper5}). Homepage/source code: \href{https://github.com/fdavidcl/ae-case-studies}{github.com/fdavidcl/ae-case-studies}.
    \item Reducing complexity (associated to \autoref{ch:paper6}). Homepage: \href{https://ari-dasci.github.io/S-reducing-complexity/}{ari-dasci.github.io/S-reducing-complexity}. Source code: \href{https://github.com/ari-dasci/S-reducing-complexity}{github.com/ari-dasci/S-reducing-complexity}.
    \item Convolutional Slicer (associated to \autoref{ch:paper7}). Homepage/source code: \href{https://github.com/fdavidcl/slicer-conv}{github.com/fdavidcl/slicer-conv}.
\end{itemize}

\subsection{Collaborations and other related results}

Articles published in collaboration with other researchers with tangential topics to the current thesis:

\begin{itemize}
    \item Charte, F., Rivera, A. J., Charte, D., del Jesus, M. J., \& Herrera, F. (2018). Tips, guidelines and tools for managing multi-label datasets: The mldr.datasets R package and the Cometa data repository. Neurocomputing, 289, 68-85.
    \item Górriz, J. M., Ramírez, J., Ortíz, A., Martinez-Murcia, F. J., Segovia, F., Suckling, J., \dots \& Ferrández, J. M. (2020). Artificial intelligence within the interplay between natural and artificial computation: Advances in data science, trends and applications. Neurocomputing, 410, 237-270.
    \item Tabik, S., Gómez-Ríos, A., Martín-Rodríguez, J. L., Sevillano-García, I., Rey-Area, M., Charte, D., ... \& Herrera, F. (2020). COVIDGR dataset and COVID-SDNet methodology for predicting COVID-19 based on chest X-ray images. IEEE Journal of biomedical and health informatics, 24(12), 3595-3605.
    \item Pascual-Triana, J. D., Charte, D., Andrés Arroyo, M., Fernández, A., \& Herrera, F. (2021). Revisiting data complexity metrics based on morphology for overlap and imbalance: snapshot, new overlap number of balls metrics and singular problems prospect. Knowledge and Information Systems, 63(7), 1961-1989.
    \item Luengo, J., Moreno, R., Sevillano, I., Charte, D., Peláez-Vegas, A., Fernández-Moreno, M., ... \& Herrera, F. (2022). A tutorial on the segmentation of metallographic images: Taxonomy, new MetalDAM dataset, deep learning-based ensemble model, experimental analysis and challenges. Information Fusion, 78, 232-253.
\end{itemize}

Educational/training material:

\begin{itemize}
    \item Charte, Francisco \& Charte, David (2021). Machine Learning y Ciencia de Datos con Python y R. Krasis Consulting. ISBN: 978-8494582257.
    \item Curso Math-ML Módulo 2: Álgebra lineal y reducción de la dimensionalidad. Published in collaboration with the Andalusian Research Institute in Data Science and Computational Intelligence (DaSCI). Video playlist: \href{https://www.youtube.com/playlist?list=PL88MWrW4s4nf-Bc3hccxt3Att8TSS-LBn}{youtube.com/ playlist?list=PL88MWrW4s4nf-Bc3hccxt3Att8TSS-LBn}.
\end{itemize}

% \setchapterpreamble[u]{\margintoc}
\section{Future lines of work}
% \label{ch:future}

\subsection{Label separability in multi-label data}

\subsection{Promoting other behavior in learned representations}

% \section{}

% \pagelayout{margin}
% \appendix % From here onwards, chapters are numbered with letters, as is the appendix convention

% \makepart{Appendix}

% \chapter{Some more blindtext}

% \blindtext

%----------------------------------------------------------------------------------------

\backmatter % Denotes the end of the main document content
\setchapterstyle{plain} % Output plain chapters from this point onwards

%----------------------------------------------------------------------------------------
%	BIBLIOGRAPHY
%----------------------------------------------------------------------------------------

% The bibliography needs to be compiled with biber using your LaTeX editor, or on the command line with 'biber main' from the template directory

\defbibnote{bibnote}{Here are the references in citation order.\par\bigskip} % Prepend this text to the bibliography
\printbibliography[heading=bibintoc, title=Bibliography, prenote=bibnote] % Add the bibliography heading to the ToC, set the title of the bibliography and output the bibliography note

%----------------------------------------------------------------------------------------
%	INDEX
%----------------------------------------------------------------------------------------

% The index needs to be compiled on the command line with 'makeindex main' from the template directory

\printnomenclature % Output the index

\end{document}
